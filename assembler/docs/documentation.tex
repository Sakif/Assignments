\documentclass[titlepage, 12pt]{article}
\usepackage[letterpaper, margin=1in]{geometry}
\usepackage{xcolor, hyperref}
\pagecolor{black}
\color{white}
\pdfminorversion=7
\renewcommand{\familydefault}{\sfdefault}
\hypersetup{bookmarks=false, unicode=true, pdftoolbar=false, pdfmenubar=false, pdfstartview={FitH}, pdftitle={XM2}, pdfauthor={Sakif Fahmid Zaman}, pdfsubject={Computer Architecture}, colorlinks=true}

\title{Design Document for XM2}
\author{Sakif Fahmid Zaman}
\begin{document}
\maketitle\tableofcontents\clearpage
\section{Introduction}
Design document for XM2.

\subsection{Basic concept}
The assemble is to do the following process one after the other:
\begin{enumerate}
	\item open an ams file
	\item read the file one line at a time
	\item remove comments form the line
	\item create tokens words on whitespace
	\item check if first token is instruction or directive
	\item IF token is instruction or directive then check if the second (last) token contains correct number and correct type of operands
	\item ELSE check if first token is label, check if label has duplicate, update table if symbol is unknown
	\item second token must be instruction or directive
	\item if second token is instruction or directive ensure the third (last) token contains the correct number and correct type of operands
\end{enumerate}

\section{Data Dictionary}
There are several item is the assembler process that require a data dictionary definition.
\subsection{Record}
Each line in the assembly file is a record. A record contains the following:

record = (label) + ([instruction | directive] + (operands))

\subsection{Symbols}
Registers and labels are stored in the symbol table. Each symbol had the following information:

symbol = name + [UNKNOWN | REGISTER | LABEL] + value

where

name = 1\{alphabetic\}1 + 0\{alphanumeric\}31

The length of the symbol name cannot be more than 32 characters long, and

alphabetic = [\_ | a-z | A-Z]

alphanumeric = [\_ | a-z | A-Z | 0-9]

\subsection{Instruction/Directive Table}
The instructions and directives can be stores in a table with each record containing the following:

ISTDID = name + operands + (byte code)

operands = [INSTRUCTION | DIRECTIVE]

To assist with searching a NONE can be added to type.

\subsection{Assembler}
The assembler must contain the following items:

Assembler = asmFile + lisFile + lineNumber + programCounter + errorCounter + symbolTable

\section{Block Diagram}

\end{document}